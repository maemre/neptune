\section{Neptune} \label{neptune}

Neptune is our parallel answer to the Julia's single-threaded garbage collector.
We use a fixed number of threads for marking, while, for reasons to be explained shortly, a single thread for sweeping.
Its structure is highly similar, overall, to Julia's, for several reasons.
First, our goal was to test the effects of parallelism on garbage collection, not to experiment with novel approachs to the general garbage collection problem.
Second, Julia's C implementation is highly complex, relying on delicate memory layout requirements, among other things.
We did not want to make our job of understanding, tracking, marking, and sweeping the contents of memory as organized originally by Julia any harder by attempting optimizations on their algorithm, apart from the complexities of adding parallelism.

\subsection{Why Rust}
Rust is a modern systems programming language, created in 2010.
It prides itself in providing many ``zero-cost'' abstractions, abstractions and guarantees that in other systems language normally require extensive library support or checks.
It gives the user memory safety and automatic memory management \textbf{without} garbage collection due to its sophisticated type system, which incorporates ``affine'' types.
All of this makes concurrency quick and easy in Rust, and also importantly, it proves a nice foreign function interface, which would be essential for interacting with Julia.

\subsection{Design Decisions}
Once our implementation of Julia's garbage collector was implemented in Rust (which took the bulk of our time), we had to decide where to proceed in achieving our goal of a parallel garbage collector.
Our general design was inspired by \cite{marlow2008parallel} and \cite{marlow2011multicore}, especially the parts about thread-local data structures and handling parallelism.
Our specific design decisions were primarily profile-guided, meaning we used the output of tools like Valgrind and OProfile, as well as in-code timers and measurements to determine where to prioritize parallelization implementation.
We saw that while marking would benefit greatly from having multiple threads, parallel sweeping actual hindered performance.
Since sweeping is mostly memory-bound, parallelizing it increased cache misses.
We also took advantage of Rust's libaries in heavily using lock-free data structures and caches in many places to prevent unnecessary blocking and lock contention.

Neptune reads value of the environment variable \texttt{NEPTUNE\_THREADS} and creates a work-stealing thread pool.
The parallel marking algorithm described in section~\ref{marking_algo} uses this thread pool rather than creating new threads every single time to avoid overhead of thread creation.

\subsection{The parallel marking algorithm}
\label{marking_algo}
We adapted Julia's marking algorithm to be multithreaded.
The original marking algorithm is implemented as a recursive algorithm that walks the root set and pushes all the leaves to a stack (called the \emph{mark stack}) when it reaches a certain recursion depth.
To parallelize this algorithm, we made a couple changes:
Firstly, we changed the mark stack with a Treiber stack, a simple lock-free, thread-safe, and memory-safe stack algorithm.
Secondly, we changed all marking with atomic updates to be thread-safe.
Finally, we implemented some thread-local caches that will get synchronized after marking to prevent having lots of expensive accesses to global data structures accompanied with locks.
To do so, we extended Julia's mark caches and added one thread-local mark cache per garbage collection worker thread.
Our thread-local mark caches contain local updates to: remsets, big object lists, and statistics for future collection decisions.

Our marking algorithm gives marking jobs for all objects in the mark stack to the thread pool and waits for the worker threads to finish and synchronize.
If the mark stack is not empty because worker threads added more objects in the meantime, the algorithm will do more iterations of the same job assignment until the mark stack is empty after synchronization.

%%% Local Variables:
%%% mode: latex
%%% TeX-master: "report"
%%% End:
